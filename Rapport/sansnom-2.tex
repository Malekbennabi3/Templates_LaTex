
\documentclass{report}
\usepackage[utf8]{inputenc}
\usepackage{graphicx}
\usepackage[Lenny]{fncychap}


\begin{document}

\Large
\tableofcontents
\newpage
\listoffigures
\newpage

\Huge


\topskip0pt
\vspace*{\fill}
'' Les opérations maintiennent les lumières allumées, la stratégie fournit une lumière au bout du tunnel, mais la gestion de projet est la locomotive qui fait avancer l’organisation.''
\newline
\newline
 Joy Gumz
\vspace*{\fill}





\large
\newpage
\chapter{Définition de Hadoop }
La bibliothèque logicielle Hadoop est un framework libre et open source écrit en Java destiné à faciliter la création d'applications distribuées (au niveau du stockage des données et de leur traitement) et échelonnables permettant aux applications de travailler avec des milliers de nœuds et des pétaoctets de données à l'aide de modèles de programmation simples. Ainsi chaque nœud est constitué de machines standard regroupées en grappe. Tous les modules de Hadoop sont conçus selon l'idée que les pannes matérielles sont fréquentes et qu'en conséquence elles doivent être gérées automatiquement par le framework.
  Elle est conçue pour passer d'un simple serveur à des milliers de machines, chacune offrant des capacités de calcul et de stockage locales. Plutôt que de s'appuyer sur le matériel pour assurer une haute disponibilité, la bibliothèque elle-même est conçue pour détecter et gérer les défaillances au niveau de la couche applicative, offrant ainsi un service hautement disponible au-dessus d'une grappe d'ordinateurs, dont chacun peut être sujet à des défaillances.
Hadoop a été inspiré par la publication de MapReduce, GoogleFS et BigTable de Google. Hadoop a été créé par Doug Cutting et fait partie des projets de la fondation logicielle Apache depuis 2009.




\Large
\vspace*{\fill}
\section*{Architecture de Hadoop}

\begin{itemize}
 \item NameNode: L'espace de noms HDFS est une hiérarchie de fichiers et de répertoires. Les fichiers et les répertoires sont représentés sur le NameNode par des inodes, qui enregistrent des attributs tels que les permissions, les temps de les temps de modification et d'accès, l'espace de noms et les quotas d'espace disque. 
 Le contenu des fichiers est divisé en grands blocs (généralement 128 mégaoctets, mais l'utilisateur peut choisir fichier par fichier) et chaque bloc du fichier est répliqué de manière indépendante sur plusieurs DataNodes (généralement trois, mais l'utilisateur peut choisir fichier par fichier).
mais l'utilisateur peut choisir de le faire fichier par fichier). Le NameNode maintient l'arborescence de l'espace de noms et la correspondance entre les blocs de fichiers et les DataNodes.
(l'emplacement physique des données du fichier). Un client HDFS souhaitant
lire un fichier contacte d'abord le NameNode pour connaître l'emplacement des blocs de données qui composent le fichier, puis lit le contenu des blocs à partir du NameNode.
blocs de données composant le fichier, puis lit le contenu des blocs à partir du
le DataNode le plus proche du client. Lorsqu'il écrit des données, le client demande au NameNode de désigner une suite de trois DataNodes pour héberger la réplication des blocs.
DataNodes pour héberger les répliques des blocs. Le client écrit ensuite
les données aux DataNodes selon le principe du pipeline. La conception actuelle
comporte un seul NameNode pour chaque cluster. Le cluster peut avoir
des milliers de DataNodes et des dizaines de milliers de clients HDFS
par cluster, car chaque DataNode peut exécuter plusieurs tâches d'application
plusieurs tâches d'application simultanément.
HDFS conserve l'ensemble de l'espace de noms dans la mémoire vive. Les données relatives aux inodes
et la liste des blocs appartenant à chaque fichier constituent les métadonnées du système de noms appelé image. L'enregistrement persistant
de l'image stockée dans le système de fichiers natif de l'hôte local est
appelé point de contrôle. Le NameNode stocke également le journal des modifications de l'image, appelé journal, dans le système de fichiers natif de l'hôte local. Pour améliorer la durabilité, des copies redondantes du point de contrôle et du journal peuvent être effectuées.
redondantes du point de contrôle et du journal peuvent être effectuées sur d'autres serveurs. Lors des redémarrages, le NameNode restaure l'espace de noms en lisant l'espace de noms et en rejouant le journal.
en lisant l'espace de noms et en rejouant le journal. L'emplacement des
répliques de blocs peuvent changer au fil du temps et ne font pas partie du point de contrôle persistant.
point de contrôle persistant.



\item Dans le Chapitre 2 :Ce chapitre présentera notre application web, nous débuterons avec la conception en utilisant le formalisme UML, nous aborderons aussi les différentes fonctionnalités de celle-ci et nous finirons par la présentation de quelques interfaces.

\end{itemize}

\vspace*{\fill}
\newpage






\chapter{ Les méthodes de Conception } 

\huge
'' Une petite impatience ruine un grand projet.'' \textbf{– Confucius.} \newline

\Large
\section{Introduction du Chapitre }


Dans les grands projets en informatique la planification est une tache cruciale et très importante, le chef de projet dois savoir comment planifier, distribuer et piloter son équipe à la lettre car la réussite du projet repose sur les stratégies et les méthodes que dois entreprendre celui-ci, il doit organiser parfaitement les différentes phases du projet afin de garantir la qualité du livrable, le respect des délais et la maîtrise des ressources.

La planification d’un projet doit être réalisée avec autant de soin, c’est qu’elle va décider de son déroulement par la suite.

\newpage
\huge
''Être un chef de projet, c’est comme être un artiste, vous avez les différents flux de processus colorés qui se combinent en une œuvre d’art.''
\textbf{-Greg Cimmarrusti.}

\Large
\section{Définition de La planification}
La gestion de projet revient à jongler avec trois balles : le temps, le coût et la qualité.
\newline
En gestion de projet, la planification est une étape incontournable et essentielle pour réussir vos projets. Pour résumer, il s'agit d'identifier les tâches à réaliser, de les hiérarchiser, de définir leur durée et leur échéance et de leur attribuer des ressources. Le but est d'établir le calendrier du projet
\newline
\newline
\textbf{Qui Est responsable de la planification ?}
Celui qui est responsable de la planification du début jusqu’à la fin c’est le chef de projet, son rôle et d’organiser et de conduire le projet de bout en bout, il assume la responsabilité du déroulement de chaque phase depuis la traduction des besoins utilisateurs en spécifications fonctionnelles et techniques, jusqu'à la recette utilisateur, voire la mise en production.





\large
 Le chef de projet doit posséder une vision pertinente du but vers lequel il souhaite emmener le projet et doit aussi avoir la capacité de l'articuler avec souplesse.
Il est capable d'anticiper les changements et d'être en mesure de déterminer, si nécessaire, de nouvelles limites au projet.
\newpage
\huge
\topskip0pt
\vspace*{\fill}
''Les chefs de projet fonctionnent comme des chefs d’orchestre qui rassemblent leurs joueurs, chacun étant un spécialiste avec une partition individuelle et un rythme interne. Sous la direction du chef, ils répondent tous au même rythme.''
\newline
\newline
\textbf{L.R. Sayles}
\vspace*{\fill}



\Large
\newpage
\section{Les Méthode de planification}

Il existe plusieurs méthodes afin de faire la planification des projets informatique, Choisir une méthode adéquate dépend du type du projet que l’équipe de développement vas entamer, Nous allons présenter ici douze méthodes de planification, afin de vous aider à sélectionner celle qui conviendra le mieux à votre projet.
Voici quelque méthodes utilisées pour la planification:

\begin{enumerate}

\item Traditionnelle
\item Agile
\item Cascade
\item PERT
\item SCRUM
\item Crystal clear
\item Processus Unifié 
\item Adaptative
\end{enumerate}


Afin de réaliser notre projet du module génie logiciel nous avons choisi d’étudier l'une des méthodes les plus utilisées pour la réalisation des projets, la méthode PERT
Pour cela voici une petite définition de cette méthode.

\newpage
\section{Définition de la méthode PERT}
PERT (en anglais : Program Evaluation and Review Technique) est une méthode conventionnelle utilisable en gestion de projet, ordonnancement et planification développée aux États-Unis par la Navy dans les années 1950.
Elle fournit des méthodes et des moyens logiques pour analyser et ordonnancer les taches pour enfin être représenter à travers un graphe qui est le diagramme ou le réseau de PERT, celui-ci représente le planning des tache et l’interconnexion entre chacune D’elle.

\vspace*{\fill}

\section{Le But de la méthode PERT}
Le but de la méthode PERT est de trouver la meilleure organisation possible pour qu'un projet soit terminé dans les meilleurs délais, et d'identifier les tâches critiques, c'est-à-dire les tâches qui ne doivent souffrir d'aucun retard sous peine de retarder l'ensemble du projet.

\vspace*{\fill}

\section{Conclusion Du chapitre}
Ce Chapitre était une présentation générale De la philosophie de gestion de projet, ainsi nous avons aborder la méthode PERT qui est très connue et utilisée pour planifier les grands projets informatiques.
Le chapitre suivant abordera le coter conceptuel de notre projet ainsi que les différente fonctionnalités fournis par notre application.

\vspace*{\fill}

\chapter{Niveau Conceptuel}


\section{Introduction du chapitre}
Ce chapitre présentera le niveau conceptuel de l’application, pour cela nous avons utilisé le formalisme UML pour la conception en se basant sur 3 diagrammes : diagramme de cas d’utilisation qui représentera l’interaction des acteurs avec notre système, le diagramme de classe qui vas représenter notre base de données, nous finirons avec le diagramme de séquence qui montrera la chronologie de quelques taches , nous allons aussi aborder les outils que nous avons utilisés pour la réalisation de notre application web .

\vspace*{\fill}

\section{Le But de Notre application}
Notre application web aura pour but d’aider le chef de projet au cours de la planification, Le processus de la méthode PERT sera automatisé et facile à manipuler

\vspace*{\fill}

\newpage
\section{Les fonctionnalités de notre application}

\begin{enumerate}

\item Nous Allons fournir au chef de projet une interface d’authentification pour accéder à l’application
\item  Le chef de projet aura son propre WorkSpace ou il pourra décomposer le projet en cours d’exécution en taches
\item Le Chef de projet pourra attribuer une durée adéquate a chaque tâche du projet et les ressources nécessaires pour sa réalisation
\item Le chef de projet pourra consulter l’avancement du projet à travers le diagramme PERT qui sera généré
\item Le chef de projet pourra aussi consulter l’avancement de chaque équipe

\end{enumerate}

\vspace*{\fill}

\section{Partie Conception}
Pour la conception de notre application nous allons utiliser 3 diagrammes connus du formalisme UML, nous débuterons d’abord par la vue utilisateur qui sera représentée par le diagramme de cas d’utilisation et la base de données qui sera représentée par le diagramme de classe, nous finirons ensuite par le diagramme de séquence qui vas représenter la synchronisation des fonctionnalités.

\vspace*{\fill}

\begin{itemize}

\newpage
\item \textbf{ Le diagramme de cas d’utilisation :}

\begin{figure}[!ht] 
    \center 
    \includegraphics[width=1\linewidth]{C:/Users/HP/Downloads/cu.png} 
    \caption{Le diagramme des cas d’utilisation }
\end{figure}

\vspace*{\fill}

\textbf{Les Acteurs principaux :}
\begin{itemize} 
\item Le chef de projet : c’est celui qui va planifier le projet du début jusqu’à la fin 
\item Le chef d’équipe : c’est le responsable de chaque équipe, celui qui contrôle le taux d’avancement de l’équipe  
\item Le maître d’ouvrage (acteur secondaire) : il peut contrôler le taux d’avancement de son projet 
\end{itemize}
\vspace*{\fill}
\newpage

\vspace*{\fill}

\item \textbf{Diagramme de classe : }
Représente les différentes tables présentes dans la base de données.

\begin{figure}[!ht] 
    \center 
    \includegraphics[width=1\linewidth]{C:/Users/HP/Downloads/dc.png} 
    \caption{Le diagramme de classe }
\end{figure}

\vspace*{\fill}

\newpage
\item \textbf{Diagramme de séquence }

\begin{figure}[!ht] 
    \center 
    \includegraphics[width=1\linewidth]{C:/Users/HP/Downloads/ddc.png} 
    \caption{Le diagramme de séquences }
\end{figure}

\end{itemize}


\vspace*{\fill}

\huge
\chapter*{Conclusion Générale}
\addcontentsline{toc}{chapter}{Conclusion Générale} 

La réalisation de ce projet nous a permis de mieux maîtriser les nouvelles technologies du web, et ainsi de travailler avec une équipe très active et collaborative, ça nous a permis aussi de nous développer en terme de communication et de conception, et de constater l’importance d’une bonne planification du projet a l’aide de la méthode PERT.

\vspace*{\fill}


\end{document}
